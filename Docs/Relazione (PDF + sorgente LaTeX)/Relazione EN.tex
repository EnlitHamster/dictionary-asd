\documentclass[a4paper,11pt]{article}
\usepackage[T1]{fontenc}
\usepackage[utf8]{inputenc}
\usepackage{lmodern}
\usepackage[english]{babel}

\title{\textit{Report exercise 2}}
\author{Daniele Liberatore, Sandro Massa, Matteo Palazzo}

\begin{document}
	
\maketitle

\section*{Edit-distance's functioning}
For the dynamic programming implementation, we save the carried out operations to avoid redundancies. Those are due to the sub-problems not being independent from one another, making this technique, known as \textit{memoization}, very effective.\\
The length of the strings \textit{X} and \textit{Y} are the dimensions of the matrix \textit{c} containing the results. The cell $c_{ij}$ stores the edit-distance from the prefix $X_{i}$ to $Y_{j}$. Initially all cells are empty; they will store the result of the first time the corresponding operation is called. Without \textit{memoization} the algorithm equals its recursive version (see slides for reference).\\
Below are some comparisons between the two versions:

\begin{table}[h!]
	\centering\small
	\begin{tabular}{c|c|c|c|c}
		\multicolumn{1}{c}{\textit{X}}
		&\multicolumn{1}{c}{\textit{Y}}
		&\multicolumn{1}{c}{\textit{Edit-distance}}
		&\multicolumn{1}{c}{\textit{Recursive}}
		&\multicolumn{1}{c}{\textit{Dynamic}}\\ \hline
		Calla                 &Palla                &1   &0.100s   &0.100s\\
		Bcamap                &Palla                &5   &0.100s   &0.100s\\
		Lonpirante            &Lungimirante         &4   &0.800s   &0.100s\\
		Antocartistuzinale    &Anticostituzionale   &5   &> 30m    &0.100s\\
	\end{tabular}
	%   \caption{Comparazioni}
\end{table}

\section*{Checker's functioning}
\textit{Checker} compares each word from \textit{target} (correctme.txt) with each one in \textit{dict} (dictionary.txt), associating to each word of the former a list of words of the latter with minimum edit-distance. If a \textit{dict} word's edit-distance is higher than the one already calculated, the former is ignored.\\
\textit{Checker\_Test} terminates in an average of 5 seconds; performances increase if the \textit{dict} is ordered by increasing length of its words. Complexity however changes only by a constant factor.
	
\end{document}

